\documentclass[a4paper,10pt]{article}
\usepackage[utf8]{inputenc}

%opening
\title{Primer archivo de LaTex} % Cambia el titulo por alguno que tu elijas.
\author{Iker Díaz Hernández} % Cambialo por tu nombre completo.

\begin{document}

\maketitle

% \begin{abstract}
% 
% \end{abstract}

\section{Seccion 1}

Hola profes de Computación 8093. 

Este es mi primer documento y emociona aprender a usar LaTex, ya que es una herramienta muy útil para la creación de textos científicos y técnicos que tengan lenguaje matemático.
\\

LA IDENTIDAD DE EULER \\
\\
\begin{center}
$ e^{i \pi} + 1 = 0 $ \\
\end{center}

Una de las fórmulas más elegantes de las matemáticas, deducida por Leonnard Euler, su belleza aflora cuando notamos que aparecen 5 números fundamentales en el mundo de las matemáticas y principalmente es utilizada para relacionar la trigonometría con el análisis matemático.\\
\\
LA SUMA DE GAUSS\\
\\
\begin{center}
$\displaystyle\sum_{k=1}^{n}k=\frac{n(n+1)}{2}$\\
\end{center}

Cuenta la leyenda que cuando Gauss era un niño, el profesor puso a su grupo a sumar cada número consecutivo del 1 al 100, lo que normalmente llevaría mucho tiempo, al joven Gauss le tomó solo unos instantes, ya que se dió cuenta de que sumando los extremos de los números del 1 al 100, siempre daban una cantidad de 101, al haber 50 pares que nos arrojan esta cantidad, su multiplicación nos da 5050 que impresionantemente, es lo mismo que sumar todos los números del 1 al 100, que posteriormente formalizó en esta famosa suma, lo que impresionó al profesor y posteriormente al mundo entero, no se sabe si la leyenda es verdadera pero tratándose de un genio como Gauss, todo es posible.\\

\end{document}
